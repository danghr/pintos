% VLDB template version of 2020-03-05 enhances the ACM template, version 1.7.0:
% https://www.acm.org/publications/proceedings-template
% The ACM Latex guide provides further information about the ACM template

\documentclass[sigconf, nonacm, balance=false, urlbreakonhyphens=true]{acmart}
\usepackage{amsmath, amsfonts, amsthm}
\usepackage{url}
\usepackage{makecell}
\usepackage{multirow}
\usepackage{longtable}
\usepackage{verbatim}
\usepackage{hyperref}

% Allow URL Breakline
\def\UrlBreaks{\do\A\do\B\do\C\do\D\do\E\do\F\do\G\do\H\do\I\do\J
\do\K\do\L\do\M\do\N\do\O\do\P\do\Q\do\R\do\S\do\T\do\U\do\V
\do\W\do\X\do\Y\do\Z\do\[\do\\\do\]\do\^\do\_\do\`\do\a\do\b
\do\c\do\d\do\e\do\f\do\g\do\h\do\i\do\j\do\k\do\l\do\m\do\n
\do\o\do\p\do\q\do\r\do\s\do\t\do\u\do\v\do\w\do\x\do\y\do\z
\do\.\do\@\do\\\do\/\do\!\do\_\do\|\do\;\do\>\do\]\do\)\do\,
\do\?\do\'\do+\do\=\do\#}

\begin{document}
    \title{CS 130 Project 4: File Systems\\Design Document}

    %%
    %% The "author" command and its associated commands are used to define the authors and their affiliations.
    \author{Tianyi Zhang}
    \affiliation{%
        \state{2018533074}
    }
    \email{zhangty2@shanghaitech.edu.cn}

    \author{Haoran Dang}
    \affiliation{%
        \state{2018533259}
    }
    \email{danghr@shanghaitech.edu.cn}

    \author{Derun Li}
    \affiliation{%
        \state{2018533152}
    }
    \email{lidr@shanghaitech.edu.cn}

    \maketitle

    \setcounter{section}{-1}

    \section{Preliminaries}
        %%
        %% If you have any preliminary comments on your submission, notes for the
        %% TAs, or extra credit, please give them here.
        \subsection{Preliminary Comments}

        No preliminary comment for this project. 

        %%
        %% Please cite any offline or online sources you consulted while
        %% preparing your submission, other than the Pintos documentation, course
        %% text, lecture notes, and course staff.
        \subsection{References}
        
            \begin{itemize}
                \item Pintos Guide by Stephen Tsung-Han Sher:  \url{https://static1.squarespace.com/static/5b18aa0955b02c1de94e4412/t/5e1bb4809e4b0a78012be132/1578874001361/Sher\%282016%29_Pintos_Guide}
            \end{itemize} 
    
    \section{Indexed and Extensible Files}
            
        \label{Indexed and Extensible Files}
    
        \subsection{Data Structures}
        
            \subsubsection{Copy here the declaration of each new or changed `\texttt{struct}' or `\texttt{struct}' member, global or static variable, `\texttt{typedef}', or enumeration. Identify the purpose of each in 25 words or less. }
        
                \begin{itemize}
                    \item In file \texttt{}
\begin{verbatim}
\end{verbatim}
                \end{itemize}
            
            \subsubsection{What is the maximum size of a file supported by your inode structure?  Show your work. }
        
        \subsection{Synchronization}

            \subsubsection{Explain how your code avoids a race if two processes attempt to extend a file at the same time. }

            \subsubsection{Suppose processes A and B both have file F open, both positioned at end-of-file.  If A reads and B writes F at the same time, A may read all, part, or none of what B writes.  However, A may not read data other than what B writes, e.g. if B writes nonzero data, A is not allowed to see all zeros.  Explain how your code avoids this race. }

            \subsubsection{Explain how your synchronization design provides "fairness".  File access is "fair" if readers cannot indefinitely block writers or vice versa.  That is, many processes reading from a file cannot prevent forever another process from writing the file, and many processes writing to a file cannot prevent another process forever from reading the file. }
        
        \subsection{Rationale}
            
            \subsubsection{Is your \texttt{inode} structure a multilevel index?  If so, why did you choose this particular combination of direct, indirect, and doubly indirect blocks?  If not, why did you choose an alternative \texttt{}{inode} structure, and what advantages and disadvantages does your structure have, compared to a multilevel index? } 
    
    \section{Subdirectories}

        \label{Subdirectories}

        \subsection{Data Structures}
        
            \subsubsection{Copy here the declaration of each new or changed `\texttt{struct}' or `\texttt{struct}' member, global or static variable, `\texttt{typedef}', or enumeration. Identify the purpose of each in 25 words or less. }
    
                \begin{itemize}
                    \item In file \texttt{}
\begin{verbatim}
\end{verbatim}
                \end{itemize}

        \subsection{Algorithms}

            \subsubsection{Describe your code for traversing a user-specified path.  How do traversals of absolute and relative paths differ? } 
        
        \subsection{Synchronization}

            \subsubsection{How do you prevent races on directory entries?  For example, only one of two simultaneous attempts to remove a single file should succeed, as should only one of two simultaneous attempts to create a file with the same name, and so on. }

            \subsubsection{Does your implementation allow a directory to be removed if it is open by a process or if it is in use as a process's current working directory?  If so, what happens to that process's future file system operations?  If not, how do you prevent it? }
        
        \subsection{Rationale}

            \subsubsection{Explain why you chose to represent the current directory of a process the way you did. }
    
    \section{Buffer Cache}

        \label{Buffer Cache}

        \subsection{Data Structures}
            
            \subsubsection{Copy here the declaration of each new or changed `\texttt{struct}' or `\texttt{struct}' member, global or static variable, `\texttt{typedef}', or enumeration. Identify the purpose of each in 25 words or less. }
    
            \begin{itemize}
                \item In file \texttt{}
\begin{verbatim}
\end{verbatim}
            \end{itemize}

        \subsection{Algorithms}

            \subsubsection{Describe how your cache replacement algorithm chooses a cache block to evict. } 

            \subsubsection{Describe your implementation of write-behind. }

            \subsubsection{Describe your implementation of read-ahead. }
        
        \subsection{Rationale}

            \subsubsection{When one process is actively reading or writing data in a buffer cache block, how are other processes prevented from evicting that block? }

    \section{Survey Questions}

        % Answering these questions is optional, but it will help us improve the
        % course in future quarters.  Feel free to tell us anything you
        % want--these questions are just to spur your thoughts.  You may also
        % choose to respond anonymously in the course evaluations at the end of
        % the quarter.

        \subsubsection*{In your opinion, was this assignment, or any one of the three problems in it, too easy or too hard? Did it take too long or too little time? }

        \subsubsection*{Did you find that working on a particular part of the assignment gave you greater insight into some aspect of OS design? }

        \subsubsection*{Is there some particular fact or hint we should give students in future quarters to help them solve the problems? Conversely, did you find any of our guidance to be misleading? }

        \subsubsection*{Do you have any suggestions for the TAs to more effectively assist students, either for future quarters or the remaining projects? }

        \subsubsection*{Any other comments? }
    
    \section*{Contributors}

        \begin{center}
            \begin{tabular}{|c|c|c|c|c|}
                \hline
                \multicolumn{2}{|c|}{Task} & \makecell{Tianyi\\Zhang} & \makecell{Haoran\\Dang} & \makecell{Derun\\Li} \\
                \hline
                \multirow{4}{*}{\makecell{Task 1 \\ Indexed and \\ Extensible \\ Files}} 
                    & Concept & \checkmark & \checkmark & \checkmark \\
                    \cline{2-5}
                    & Implementation & \checkmark & \checkmark & \checkmark \\
                    \cline{2-5}
                    & Debugging & \checkmark & \checkmark & \checkmark \\
                    \cline{2-5}
                    & Design Document & \checkmark & \checkmark & \checkmark \\
                \hline
                \multirow{4}{*}{\makecell{Task 2 \\ Subdirector-\\ies}} 
                    & Concept & \checkmark & \checkmark & \checkmark \\
                    \cline{2-5}
                    & Implementation & \checkmark & \checkmark & \checkmark \\
                    \cline{2-5}
                    & Debugging & \checkmark & \checkmark & \checkmark \\
                    \cline{2-5}
                    & Design Document & \checkmark & \checkmark & \checkmark \\
                \hline
                \multirow{4}{*}{\makecell{Task 3 \\ Buffer \\ Cache}} 
                    & Concept & \checkmark & \checkmark & \checkmark \\
                    \cline{2-5}
                    & Implementation & \checkmark & \checkmark & \checkmark \\
                    \cline{2-5}
                    & Debugging & \checkmark & \checkmark & \checkmark \\
                    \cline{2-5}
                    & Design Document & \checkmark & \checkmark & \checkmark \\
                \hline
            \end{tabular}
        \end{center}

\end{document}
\endinput
