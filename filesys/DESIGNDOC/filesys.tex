% VLDB template version of 2020-03-05 enhances the ACM template, version 1.7.0:
% https://www.acm.org/publications/proceedings-template
% The ACM Latex guide provides further information about the ACM template

\documentclass[sigconf, nonacm, balance=false, urlbreakonhyphens=true]{acmart}
\usepackage{amsmath, amsfonts, amsthm}
\usepackage{url}
\usepackage{makecell}
\usepackage{multirow}
\usepackage{longtable}
\usepackage{verbatim}
\usepackage{hyperref}

% Allow URL Breakline
\def\UrlBreaks{\do\A\do\B\do\C\do\D\do\E\do\F\do\G\do\H\do\I\do\J
\do\K\do\L\do\M\do\N\do\O\do\P\do\Q\do\R\do\S\do\T\do\U\do\V
\do\W\do\X\do\Y\do\Z\do\[\do\\\do\]\do\^\do\_\do\`\do\a\do\b
\do\c\do\d\do\e\do\f\do\g\do\h\do\i\do\j\do\k\do\l\do\m\do\n
\do\o\do\p\do\q\do\r\do\s\do\t\do\u\do\v\do\w\do\x\do\y\do\z
\do\.\do\@\do\\\do\/\do\!\do\_\do\|\do\;\do\>\do\]\do\)\do\,
\do\?\do\'\do+\do\=\do\#}

\begin{document}
    \title{CS 130 Project 4: File Systems\\Design Document}

    %%
    %% The "author" command and its associated commands are used to define the authors and their affiliations.
    \author{Tianyi Zhang}
    \affiliation{%
        \country{2018533074}
    }
    \email{zhangty2@shanghaitech.edu.cn}

    \author{Haoran Dang}
    \affiliation{%
        \country{2018533259}
    }
    \email{danghr@shanghaitech.edu.cn}

    \author{Derun Li}
    \affiliation{%
        \country{2018533152}
    }
    \email{lidr@shanghaitech.edu.cn}

    \maketitle

    \setcounter{section}{-1}

    \section{Preliminaries}
        %%
        %% If you have any preliminary comments on your submission, notes for the
        %% TAs, or extra credit, please give them here.
        \subsection{Preliminary Comments}

        No preliminary comment for this project. 

        %%
        %% Please cite any offline or online sources you consulted while
        %% preparing your submission, other than the Pintos documentation, course
        %% text, lecture notes, and course staff.
        \subsection{References}
        
            \begin{itemize}
                \item Pintos Guide by Stephen Tsung-Han Sher:  \url{https://static1.squarespace.com/static/5b18aa0955b02c1de94e4412/t/5e1bb4809e4b0a78012be132/1578874001361/Sher\%282016%29_Pintos_Guide}
            \end{itemize} 
    
    \section{Indexed and Extensible Files}
            
        \label{Indexed and Extensible Files}
    
        \subsection{Data Structures}
        
            \subsubsection{Copy here the declaration of each new or changed `\texttt{struct}' or `\texttt{struct}' member, global or static variable, `\texttt{typedef}', or enumeration. Identify the purpose of each in 25 words or less. }
        
                \begin{itemize}
                    \item In file \texttt{filesys/inode.c}
\begin{verbatim}
/* Number of direct blocks in an inode. */
#define DIRECT_BLOCK 12
/* Number of indirect blocks stored 
   in a sector. */
#define INDIRECT_BLOCK \
    (BLOCK_SECTOR_SIZE / \
    (sizeof (block_sector_t)))
/* Maximum number of sectors in an inode */
#define MAXIMUM_SECTORS_IN_INODE \
    DIRECT_BLOCK + INDIRECT_BLOCK + \
    INDIRECT_BLOCK * INDIRECT_BLOCK

...

/* inode operation lock. */
struct lock inode_extension_lock;

...

struct inode_disk
{
    /* Direct and indirect blocks. 
       - DIRECT_BLOCK blocks
       - 1 indirect block
       - 1 double indirect block */
    block_sector_t blocks[DIRECT_BLOCK + 2];
    
    ...

    /* To meet BLOCK_SECTOR_SIZE size 
       requirement. */
    char unused[BLOCK_SECTOR_SIZE
        - sizeof (block_sector_t) 
            * (DIRECT_BLOCK + 2)
        - sizeof (off_t)
        - sizeof (unsigned)
        ...
        ];
};


/* Indirect blocks stored in a sector. */
struct inode_indirect_block_sector
{
    /* Sectors */
    block_sector_t blocks[INDIRECT_BLOCK];
};

/* Double indirect blocks stored in a 
   sector. */
struct inode_double_indirect_block_sector
{
    /* Sectors */
    block_sector_t 
    indirect_blocks[INDIRECT_BLOCK];
}; 
\end{verbatim}
                \end{itemize}
            
            \subsubsection{What is the maximum size of a file supported by your inode structure?  Show your work. }

                In this file structure, in total $12 + 128 + 128^2 = 16524$ sectors can be stored in an inode. Since each sector contains $512$B data, the maximum file length is $512\text{B} \times 16524 = 8.068\text{MiB}$. 
        
        \subsection{Synchronization}

            \subsubsection{Explain how your code avoids a race if two processes attempt to extend a file at the same time. }

                The operation of extension of file must be done with lock \texttt{file\_extension\_lock} acquired, which avoids the race. After one have acquired a lock, it will double check whether file extension is needed, to avoid the situation where the file is extended when it is waiting for the lock. 

            \subsubsection{Suppose processes A and B both have file F open, both positioned at end-of-file.  If A reads and B writes F at the same time, A may read all, part, or none of what B writes.  However, A may not read data other than what B writes, e.g. if B writes nonzero data, A is not allowed to see all zeros.  Explain how your code avoids this race. }

                When reading data, function \texttt{inode\_read\_at} will check each byte at the very beginning, and the loop will be terminated when the offset does not exist in the sectors, so data cannot be read when not allocated in the sectors. 

            \subsubsection{Explain how your synchronization design provides "fairness".  File access is "fair" if readers cannot indefinitely block writers or vice versa.  That is, many processes reading from a file cannot prevent forever another process from writing the file, and many processes writing to a file cannot prevent another process forever from reading the file. }

                The lock only locks the file extension process. Thus, this is determined by the schedular, which currently uses round-robin method to schedule all threads, so the fairness is achieved in this method. 
        
        \subsection{Rationale}
            
            \subsubsection{Is your \texttt{inode} structure a multilevel index?  If so, why did you choose this particular combination of direct, indirect, and doubly indirect blocks?  If not, why did you choose an alternative \texttt{}{inode} structure, and what advantages and disadvantages does your structure have, compared to a multilevel index? } 

                This is a two-level index which is able to carry 8MiB file as stated in section 1.1.2. 
                
                In this design, each structure contains file metadata, 12 direct blocks, a 1-level indirect block and a double indirect block. We choose this because it is much the same as the Unix File System except for lack of triple indirect blocks, and it is adequate to handle the testcases. 
    
    \section{Subdirectories}

        \label{Subdirectories}

        \subsection{Data Structures}
        
            \subsubsection{Copy here the declaration of each new or changed `\texttt{struct}' or `\texttt{struct}' member, global or static variable, `\texttt{typedef}', or enumeration. Identify the purpose of each in 25 words or less. }
    
                \begin{itemize}
                    \item In file \texttt{inode.c}
\begin{verbatim}
struct inode
{

    ...

    /* Used to distinguish the directory and 
       file. */
    bool is_dir;
    
    ...

    /* To meet BLOCK_SECTOR_SIZE size 
       requirement. */
    char unused[BLOCK_SECTOR_SIZE
                ...
                - sizeof (bool)              
                ];
};
\end{verbatim}
                \end{itemize}

 \begin{itemize}
                    \item In file \texttt{threads.h}
\begin{verbatim}
struct thread
{

    ...

    /* Store the directory opened by current
       thread. */
    struct dir* directory;
};

\end{verbatim}
                    \end{itemize}
\begin{itemize}
    \item in file \texttt{syscall.c}
\begin{verbatim}
struct fd_entry
{

    ...

    /* Store the directory which holds the 
       file. */
    struct dir *directory;
};
\end{verbatim}
\end{itemize}
        \subsection{Algorithms}

            \subsubsection{Describe your code for traversing a user-specified path.  How do traversals of absolute and relative paths differ? } 
            
                We split the path and store the splited words. We use the words which are name of files or directories to go through until we find the file. When we handle the relative paths, we use the first sector of the directory to store the parent directory and we can solve the path include "\texttt{../}". When we handle the path include "\texttt{.}", we will reopen the directory.
                
        
        \subsection{Synchronization}

            \subsubsection{How do you prevent races on directory entries?  For example, only one of two simultaneous attempts to remove a single file should succeed, as should only one of two simultaneous attempts to create a file with the same name, and so on. } 

	            We will use a lock to make the remove, create operations atomic. And for a \texttt{inode} we  use \texttt{removed} to check whether the file has already been removed.


            \subsubsection{Does your implementation allow a directory to be removed if it is open by a process or if it is in use as a process's current working directory?  If so, what happens to that process's future file system operations?  If not, how do you prevent it? }
        
                Not allowed. We will check every file in the directory whether the file is in use or not. As long as there is one file is in use we cannot remove the directory.

        \subsection{Rationale}

            \subsubsection{Explain why you chose to represent the current directory of a process the way you did. }

                We store the directory in the \texttt{thread} struct, and  we only change the variable when we change the directory.

    \section{Buffer Cache}

        \label{Buffer Cache}

        \subsection{Data Structures}
            
            \subsubsection{Copy here the declaration of each new or changed `\texttt{struct}' or `\texttt{struct}' member, global or static variable, `\texttt{typedef}', or enumeration. Identify the purpose of each in 25 words or less. }
    
            \begin{itemize}
                \item In file \texttt{filesys/cache.c}
\begin{verbatim}
/* Size of buffer cache */
#define BUFFER_CACHE_SIZE 64
/* Period to flush all the cache into the 
   disk */
#define BUFFER_CACHE_FLUSH_INTERVAL 20

/* Entries of buffer cache */
struct buffer_cache_entry
{
    /* Whether this sector is being used */
    bool using;

    /* Information of the cache */
    /* Sector on the disk of the cached file */
    block_sector_t sector;
    /* Dirty bit */
    bool dirty;
    /* Last access time */
    int64_t access_time;

    /* Data storage for a block */
    uint8_t buffer[BLOCK_SECTOR_SIZE];
};

/* Buffer cache entries */
static struct buffer_cache_entry 
    buffer_cache[BUFFER_CACHE_SIZE];
/* Lock for buffer cache operations */
static struct lock buffer_cache_lock;
/* Flag that the buffer cache is initialzed */
bool buffer_cache_initialized = false;

/* Last time buffer cache flushed */
int64_t buffer_cache_last_flush = 30;
/* Last sector read, 0 for no need to read 
   ahead */
block_sector_t 
    buffer_cache_last_sector_loaded = 0;
\end{verbatim}
            \end{itemize}

        \subsection{Algorithms}

            \subsubsection{Describe how your cache replacement algorithm chooses a cache block to evict. } 

                This implementation uses LRU to choose the cache to evict. Each time when accessing a cache block, including reading, writing and reading ahead, the current timer ticks will be stored in \texttt{access\_time} of \texttt{buffer\_cache\_entry}. 
                
                When eviction is needed, we find the cache blcok with the smallest \texttt{access\_time} and evict it. 

            \subsubsection{Describe your implementation of write-behind. }

                A boolean variable \texttt{dirty} is set to \texttt{false} when data is freshly loaded, and only set to \texttt{true} when this part of cache is written. 

                The written data in cache are not directly written to the disk. Instead, it will be written back only during the flushing process, which will be invoked during eviction, periodically flushing and halting. 

            \subsubsection{Describe your implementation of read-ahead. }

                When reading a data, the sector read will be stored in \\ \texttt{buffer\_cache\_last\_sector\_loaded}. A thread repeatedly executing function \texttt{buffer\_cache\_period} will keep checking whether read-ahead is needed. If \texttt{buffer\_cache\_last\_sector\_loaded} is not zero, then the function will load the next sector into the cache if it is not loaded. 

                Since this function is executed in a separate thread, it will be done asynchronously only when this thread is scheduled. 
        
        \subsection{Rationale}

            \subsubsection{When one process is actively reading or writing data in a buffer cache block, how are other processes prevented from evicting that block? }

                A global lock of the whole buffer cache is implemented, and any operation on the buffer cache need to be done with the lock acquired. This prevents conflict eviction and reading/writing processes. 

    \section{Survey Questions}

        % Answering these questions is optional, but it will help us improve the
        % course in future quarters.  Feel free to tell us anything you
        % want--these questions are just to spur your thoughts.  You may also
        % choose to respond anonymously in the course evaluations at the end of
        % the quarter.

        \subsubsection*{In your opinion, was this assignment, or any one of the three problems in it, too easy or too hard? Did it take too long or too little time? }

            It is mostly okay in this project, but some of the directory persistance tests requires much time, as 

        \subsubsection*{Did you find that working on a particular part of the assignment gave you greater insight into some aspect of OS design? }

            All the three parts offer us the basic view of a file system, which greatly improves our understanding of that. Especially, the directory and indexed file tells us about the basic design structure of an inode-based file system. 

        \subsubsection*{Is there some particular fact or hint we should give students in future quarters to help them solve the problems? Conversely, did you find any of our guidance to be misleading? }

            Not yet. 

        \subsubsection*{Do you have any suggestions for the TAs to more effectively assist students, either for future quarters or the remaining projects? }

            Not yet. 

        \subsubsection*{Any other comments? }

            Not yet. 
    
    \section*{Contributors}

        \begin{center}
            \begin{tabular}{|c|c|c|c|c|}
                \hline
                \multicolumn{2}{|c|}{Task} & \makecell{Tianyi\\Zhang} & \makecell{Haoran\\Dang} & \makecell{Derun\\Li} \\
                \hline
                \multirow{4}{*}{\makecell{Task 1 \\ Indexed and \\ Extensible \\ Files}} 
                    & Concept & \checkmark & \checkmark & \checkmark \\
                    \cline{2-5}
                    & Implementation & \checkmark & \checkmark & \checkmark \\
                    \cline{2-5}
                    & Debugging & \checkmark & \checkmark & \checkmark \\
                    \cline{2-5}
                    & Design Document & & \checkmark & \\
                \hline
                \multirow{4}{*}{\makecell{Task 2 \\ Subdirector-\\ies}} 
                    & Concept & \checkmark & & \checkmark \\
                    \cline{2-5}
                    & Implementation & \checkmark & & \checkmark \\
                    \cline{2-5}
                    & Debugging & \checkmark & \checkmark & \checkmark \\
                    \cline{2-5}
                    & Design Document & \checkmark & & \checkmark \\
                \hline
                \multirow{4}{*}{\makecell{Task 3 \\ Buffer \\ Cache}} 
                    & Concept & & \checkmark & \\
                    \cline{2-5}
                    & Implementation & & \checkmark & \\
                    \cline{2-5}
                    & Debugging & \checkmark & \checkmark & \checkmark \\
                    \cline{2-5}
                    & Design Document & & \checkmark & \\
                \hline
            \end{tabular}
        \end{center}

\end{document}
\endinput
